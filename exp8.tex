\documentclass{article}
\usepackage{amsthm}
\newtheorem{theorem}{Theorem}[section]
\newtheorem{corollary}{Corollary}[theorem]
\newtheorem{lemma}[theorem]{Lemma}
\theoremstyle{definition}
\newtheorem{definition}{Definition}[section]
\begin{document}
	\section{Numbered Theorems, Definitions, Corollaries, and Lemmas}
	Theorems can easily be defined:
	\begin{corollary}
		Let $f$ be a function whose derivative exists at every point, then $f$ is a continuous function.
	\end{corollary}
	\begin{theorem}[Pythagorean Theorem]
		\label{pythagorean}
		This is a theorem about right triangles and can be summarized in the next equation:
		$x^2 + y^2 = z^2$.
	\end{theorem}
	A consequence of Theorem \ref{pythagorean} is the statement in the next corollary:
	\begin{corollary}
		There is no right triangle whose sides measure 3 cm, 4 cm, and 6 cm.
	\end{corollary}
	You can reference theorems, such as \ref{pythagorean}, when a label is assigned.
	\begin{lemma}
		Given two line segments whose lengths are $a$ and $b$ respectively, there is a real number $r$ such that $b = ra$.
	\end{lemma}
	\begin{definition}[Absolute Value Function]
		The absolute value function can be specified as a two-part definition as follows:\\
		$|x| = 
		\left\{
		\begin{array}{ll}
			x & \mbox{if } x \geq 0 \\ 
			-x & \mbox{if } x < 0 
		\end{array}
		\right.$
	\end{definition}
\end{document}